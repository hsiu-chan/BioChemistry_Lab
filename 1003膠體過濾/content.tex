\setlength{\parindent}{2em} %首行缩进

\section*{實驗目的:}
本實驗讓我們學習使用Sephadex G-50進行膠體過濾法,來將不同分子大小的蛋白質分離,運用的原理為分子篩(Molecular Sieve),並利用Bradford assay與ELISA製作雙曲線圖(OD$_{450nm}$, OD$_{595nm}$)並確認蛋白質所在的管柱。最後使用ELISA繪製標準曲線圖以推估原蛋白質的濃度與回收率。


\section*{實驗原理:}

\begin{enumerate}
  \item 運用\textbf{分子篩原理(Molecular sieve)},依流徑長短、大小分子停留的時間不同來分開分子。
  \item 運用的材料為Sephadex G-50,名稱來自separation Pharmacia dextran,是一種葡聚醣凝膠。\cite{Wikipedia_2023_Sephadex} Sephadex G-50 常用於凝膠過濾柱中,其中的G代表Gel,而數字則代表回水率,即 Sephadex G-50 每克的dry gel需要5.0克的水。\cite{Sephadex65:online}%https://www.sigmaaldrich.com/TW/en/product/sigma/g5080

  \item \textbf{分子篩管柱層析法}:如\prettyref{fig:mf}所示,膠體上有孔洞可以卡住小分子,讓小分子跑進去再出來,所以花較多的時間從管柱排出,而大分子卡不進去,因此小分子會走的比大分子慢,大分子會先從管柱排出,可藉此性質分離大小不同的分子。
\end{enumerate}

\begin{figure}[H]
\centering
\includegraphics[width=.5\textwidth]{paste_src/2023-10-22-15-08-32.png}
\caption{分子篩管柱層析法}
\label{fig:mf}
\end{figure}


\section*{實驗步驟:}

\subsection*{實驗器材}

\begin{enumerate}[label=\arabic*.]
  \item 管柱、管柱夾
  \item 水
  \item 滴管
  \item 1xPBS緩衝液(Phosphate buffered Saline)
  \item Sephadex G-50 膠體懸浮液
  \item Sample:BSA(bovine serum albumin,牛血清蛋白)+ Methyl Orange
  \item	Eppendorf、Micropipette、ELISA plate
  \item	Bradford reagent
\end{enumerate}


\subsection*{步驟}

\begin{enumerate}[label=\arabic*.]
  \item 製備管柱
  \begin{enumerate}[label=(\arabic*)]
    \item 取管柱
  
    \item 加水進管柱,確定管柱可讓緩衝液順暢通過
    \item 塞住管柱
    \item 先加入 1xPBS 緩衝液到管柱中,並維持八分滿的高度
    \item 加入均質的 Sephadex G-50 膠體懸浮液
    \item 打開管柱
    \item 持續加入並使其沉降至黑線位置左右
    \item 滴管加 PBS 緩衝液進管柱,沖散 Sephadex G-50 使之重新均勻堆疊
    \item 待上方緩衝液流下,直到管柱下方不再滴出緩衝液就可以塞住管柱
  \end{enumerate}
  \item 放入Sample
  \begin{enumerate}[label=(\arabic*)]
    \item 用micropipette吸取10μl的BSA+Methyl orange(本次實驗 sample,BSA 濃度 50\mug/\mul)
    \item 將micropipette伸進管柱內,把Sample 滴加在膠體正上方(不可碰到管壁)
    \item 小心將緩衝液加入管柱內,需注意避免將Sample沖散,使緩衝液滿溢到管柱口,並形成表面張力。
  \end{enumerate}
  \item 收集樣品
  \begin{enumerate}[label=(\arabic*)]
    \item 將eppendorf事先標號碼。
    \item 打開管柱,馬上開始收集濾出來的液體。
    \item 每 10 滴收集到一個 eppendorf,共收集10管 (每一管大約350\mul)。
  \end{enumerate}
  \item 檢測收集成果並繪製雙曲線圖
  \begin{enumerate}[label=(\arabic*)]
    \item 利用 micropipette,分別將收集到的10個 eppendorf 取各取10μl,加入 100μl Bradford reagent,放在eppendorf中,在室溫中反應3$\sim$5分鐘,使其混合均勻。
    \item 取100μl放入ELISA plate,利用OD$_{595nm}$測定。此為確認大分子於何管柱流出。
    \item 另外取 100μl 未跟Bradford reagent 反應過的原樣品液,依序注入ELISA plate,利用OD$_{450nm}$測定。此為確認小分子於何管柱流出。
    \item 將結果繪製於excel檔中。
  \end{enumerate} 
  \item 測定純化出的大分子蛋白的濃度。
  \begin{enumerate}[label=(\arabic*)]
    \item 取OD$_{595nm}$吸光值最高的原始管子,進行Bradford assay。
    \item 取八支eppendorf,分別放入BSA 0,2,4,6,8,10μl及 A(2\mul unknown), B(2\mul unknown)溶於\ce{ddH2O}至最終容量為10μl。
    \begin{table}[ht]
      \centering
      \begin{tabular}{ccc}
      \toprule
      管號&BSA \ce{(\mu l)}& \ce{ddH2O(\mu l)}\\
      \midrule
      1&0&10\\
      2&2&8\\
      3&4&6\\
      4&6&4\\
      5&8&2\\
      6&10&0\\
      \midrule
      A (2μl Unknown)&2&8\\
      B (4μl Unknown)&4&6\\
    
      \bottomrule
    \end{tabular}\end{table}
    \item 每管加入 300μl 的 Bradford Reagent。
    \item 加入後,用手指輕彈eppendorf底部,並靜置5分鐘。
    \item 自八支eppendorf各吸取150\mul 至96孔盤。
    \item 放入ELISA reader,利用OD$_{595nm}$測定。
  \end{enumerate}

  
\end{enumerate}



\section*{實驗結果及討論:}
\subsection*{結果}
\begin{enumerate}
  \item 根據\prettyref{fig:gel_filtration},膠體過濾實驗中 OD$_{595nm}$(大分子蛋白質)的峰值發生在Fraction 3,而OD$_{450nm}$(小分子蛋白質)的峰值發生在Fraction 6。
  \item 根據\prettyref{tab:BSA}, \prettyref{fig:stand_curve} ,得知BSA濃度標準曲線 $y=0.0673x+0.0301,\ R^2=0.966$,即可推算 Fraction 3 的濃度為 0.804 \mug/\mul。 
  \item Fraction 3 的層析液約為 350\mul,而 Sample BSA 濃度為 50\mug/\mul 因此 BSA 回收率為
  $$
  \frac{0.804(\mug/\mul)\times 350(\mul)}{50(\mug/\mul)\times 10(\mul)}=0.562
  $$



\end{enumerate}


  




\subsection*{實驗數據}

\begin{table}[h]
\begin{minipage}[t]{0.45\textwidth}
  \setlength{\abovecaptionskip}{0cm} % 调整caption间距
  \caption{膠體層析吸光值數據}\label{tab:gel_filtration}
  \begin{tabular}{lll}
    \toprule
    fraction no.&OD$_{595nm}$&OD$_{450nm}$\\
    \midrule
    1&1.695&0.241\\
    2&1.767&0.204\\
    3&2.399&0.183\\
    4&0.953&0.226\\
    5&1.284&0.673\\
    6&1.196&0.818\\
    7&1.196&0.229\\
    8&1.481&0.198\\
    \bottomrule
  \end{tabular}
\end{minipage}
\begin{minipage}[t]{0.45\textwidth}
  \setlength{\abovecaptionskip}{0cm} % 调整caption间距
  \caption{BSA吸光值數據}\label{tab:BSA}
  \begin{tabular}{lll}
    \toprule
    BSA (mg)&OD$_{595nm}$&raw data\\
    \midrule
    0&0&0.759\\
    2&0.22&0.979\\
    4&0.328&1.087\\
    6&0.364&1.123\\
    8&0.552&1.311\\
    10&0.736&1.495\\
    \midrule
    A(2\mul\ unknown)&0.116&0.875\\
    B(4\mul\ unknown)&0.291&1.050\\
    \bottomrule
  \end{tabular}
\end{minipage}
\end{table}

\begin{table}[ht]
  \setlength{\abovecaptionskip}{0cm} % 调整caption间距
  \setlength{\tabcolsep}{10mm}{
  \caption{BAS吸光值回歸直線與回收率} \label{tab:result_data}
  \begin{tabular}{llll}
    \toprule
    y=a+bx&\\
    \midrule
    a&0.0673\\
    b&0.0301\\
    R$^2$&0.966\\
    \midrule
    A濃度&0.638 \mug/\mul\\
    B濃度&0.969 \mug/\mul\\
    平均濃度&0.804 \mug/\mul\\
    回收率&0.562\\
    \bottomrule
  \end{tabular}}
\end{table}

\subsection*{實驗作圖}

\begin{figure}[H]
\centering
\includegraphics[width=.8\textwidth]{paste_src/2023-10-18-22-15-33.png}
\caption{膠體層析吸光值雙曲線}
\label{fig:gel_filtration}
\vspace{1em}
\includegraphics[width=.8\textwidth]{paste_src/2023-10-18-22-23-08.png}
\caption{BSA吸光值標準曲線}
\label{fig:stand_curve}
\end{figure}



\newpage
\subsection*{實驗討論:}
\begin{enumerate}[label=\arabic*.]
  \item 如\prettyref{tab:gel_filtration}, \prettyref{fig:gel_filtration},第一至第八管的順序為收集蛋白質的順序,可知第三管所收集到的大分子蛋白含量最高,第五和第六管則是收集到較多的小分子蛋白,因此可得出\textbf{小分子蛋白會在管柱內待得比較久}的結論。
  \item 老師上課有提到,兩個高峰值出現相距最好要超過兩管,才是比較好的分離,我們這次實驗大分子蛋白的峰值出現在第三管,小分子蛋白的峰值出現在第五及第六管,所以在分離方面有達到標準。
  \item BSA(大分子蛋白,\prettyref{fig:gel_filtration}中OD$_{595nm}$曲線)雖然第三管的峰值有做出來,但第一及第二管的數值偏高,造成此情況可能的原因是流速過快所導致。
  \item Methyl Orange(小分子蛋白,\prettyref{fig:gel_filtration}中OD$_{450nm}$曲線)峰值在第五及第六管,可能的原因為在放入Sample後,再加入緩衝液的力道過大,導致Sample有些微被沖散,因此離開管柱的時間拉長。
  \item \prettyref{fig:96}中E排為第一至第八管各取10\mul ,加入 100\mul\ Bradford reagent,已知 Bradford reagent 會與蛋白質結合呈現藍色。可以明顯以肉眼觀察到第三管有明顯的藍色,即第三管有最多的蛋白質。
  \item \prettyref{fig:96}圖中F排為第一至第八管各取100\mul,可以用肉眼觀察到第五及第六管明顯有收集到橘色的小分子蛋白(Methyl Orange)。
  \begin{figure}[H]
    \centering
    \includegraphics[width=.8\textwidth]{paste_src/2023-10-22-16-03-54.png}
    \caption{96孔盤}
    \label{fig:96}
  \end{figure}
  \item 蛋白質容液黏滯力較大,容易沾在 pipette tip 上導致 eppendorf 中 BSA 的量較少,ELISA 測出來的濃度偏低。而 \prettyref{tab:result_data} 中A管的濃度較B管低可能是因為 A 管的 BSA 量較少(2\mul),受到沾在 pipette tip 的蛋白質影響較大。我們仍需要加強 pipette 的使用技巧。
    
\end{enumerate}
  




\section{延伸討論}

\subsection*{Molecular Sieve(分子篩)}
Molecular Sieve(分子篩)是利用材料具有相似孔洞大小的特性,使小分子需要經過孔洞,所需要的時間較久,而大分子則會從孔洞旁經過,會較快經過材料,而達到分離不同大小分子的效果。IUPAC依據孔洞直徑的大小,將材料分為:\cite{Molecula6:online}%\url{https://en.wikipedia.org/wiki/Molecular_sieve}

\begin{itemize}
  \item Mircroporous material (<2nm)
  \item Mesoporous material (2nm~50nm)
  \item Macroporous material(>50nm)
\end{itemize}

然而更精確地來說,分離所依據的並非是分子的Molecular Weight,而是分子的Hydrodynamic Radius,所以就算是相同分子量的長鏈狀及球狀分子,其在分子篩管柱層析中的表現也不會一樣,球狀分子會需要比較久的時間才會析出。\cite{YMCSECMA7:online}
%https://ymctaiwan.com/portfolio/ymc_sec/#1607415269821-f456a30f-6b38






\bibliography{bibfile} 
\bibliographystyle{unsrt}
